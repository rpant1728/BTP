\section{Technicalities}

\subsection{Parity Games}

A \emph{parity game} is a tuple $G = (V,V_0,V_1,E,\Omega)$ where $(V,E)$ forms a directed graph
whose node set is partitioned into $V = V_0 \cup V_1$ with $V_0 \cap V_1 = \emptyset$, and
$\Omega : V \to \Nat$ is the \emph{priority function} that assigns to each node a natural number
called the \emph{priority} of the node. We assume the underlying graph to be total, i.e.\ for every
$v \in V$ there is a $w \in W$ s.t.\ $(v,w) \in E$.

We also use infix notation $vEw$ instead of $(v,w) \in E$ and define the set of all \emph{successors} of
$v$ as $vE := \{ w \mid vEw \}$, as well as the set of all \emph{predecessors} of $w$ as
$Ew := \{ v \mid vEw \}$.

The game is played between two players called $0$ and $1$ in the following way. Starting in a node
$v_0 \in V$ they construct an infinite path through the graph as follows. If the construction so far
has yielded a finite sequence $v_0\ldots v_n$ and $v_n \in V_i$ then player $i$ selects a $w \in v_nE$
and the play continues with the sequence $v_0\ldots v_n w$.

Every play has a unique winner given by the \emph{parity} of the greatest priority that occurs infinitely 
often in a play. The winner of the play $v_0 v_1 v_2 \ldots$ is player $i$ iff
$\max \{ p \mid \forall j \in \Nat \exists k \geq j:\, \Omega(v_k) = p \} \equiv_2 i$ (where $i \equiv_2 j$ holds iff $|i - j| \mod 2 = 0$). That is, player $0$ tries to make an even
priority occur infinitely often without any greater odd priorities occurring infinitely often, player
$1$ attempts the converse.

% The priorities occurring in a game are often ordered w.r.t. their usefulness to one of the players $i \in \{0, 1\}$ by the \emph{reward order $\preceq_i$} which is defined as follows:
% \begin{displaymath}
% p_1 \preceq_i p_2 \,:\iff\,\, rew_i(p_1) \leq rew_i(p_2)
% \end{displaymath}
% where $rew_i(p) := p$ if $p \equiv_2 i$ and $rew_i(p) := -p$ otherwise.
% \TODO{Paragraph nach hinten}

In the following we will restrict ourselves to finite parity games. It is easy to see that in a finite
parity game the winner of a play is determined uniquely since the range of $\Omega$ must necessarily
be finite as well. Technically, we are considering so-called max-parity games. There is also the
min-parity variant in which the winner is determined by the parity of the \emph{least} priority occuring
infinitely often. On finite graphs, though, these two games are equivalent in the sense that a max-parity
game $G = (V,V_0,V_1,E,\Omega)$ can be converted into a min-parity game $G' = (V,V_0,V_1,E,\Omega')$
whilst preserving important notions like winning regions, strategies, etc. Simply let $p$ an even upper
bound on all the priorities $\Omega(v)$ for any $v \in V$. Then define $\Omega'(v) := p - \Omega(v)$.
This construction also works the other way round, i.e.\ in order to transform a min-parity into a
max-parity game.

A \emph{strategy} for player $i$ is a partial function $\sigma: V^*V_i \to V$, s.t.\ for all sequences
$v_0 \ldots v_n$ with $v_{i+1} \in v_iE$ for all $j=0,\ldots,n-1$, and all $v \in V_i$:
$\sigma(v_0\ldots v_n) \in v_nE$. That is, a strategy for player $i$ assigns to every finite path through
$G$ that ends in $V_i$ a successor of the ending node. A play $v_0 v_1 \ldots$ \emph{conforms} to a strategy
$\sigma$ for player $i$ if for all $j \in \Nat$ we have: if $v_j \in V_i$ then 
$v_{j+1} = \sigma(v_0\ldots v_j)$.
Intuitively, conforming to a strategy means to always make those choices that are prescribed by the strategy.
A strategy $\sigma$ for player $i$ is a \emph{winning strategy} starting in some node $v \in V$ if player $i$ wins
every play that conforms to this strategy and begins in $v$. We say that player $i$ \emph{wins} the game $G$
starting in $v$ iff he/she has a winning strategy for $G$ starting in $v$.

With $G$ we associate two sets $W_0,W_1 \subseteq V$ with the following definition. $W_i$ is the set of
all nodes $v$ s.t.\ player $i$ wins the game $G$ starting in $v$. We write $W_i^G$ in order to name the parity 
game that the winning regions refer to, for example when it cannot uniquely be identified from the context. 

Clearly, we must have
$W_0 \cap W_1 = \emptyset$ for otherwise assume that there is a node $v$ such that both players $0$ and $1$
have winning strategies $\sigma_0$ and $\sigma_1$ for $G$ starting in $v$. Then there is a unique play
$\pi = v_0 v_1 \ldots$ such that $v_0 = v$ and $\pi$ conforms to both $\sigma_0$ and $\sigma_1$. It is
obtained by simply playing the game while both players perform their choices according to their respective
strategies. However, by definition $\pi$ is won by both players, and therefore the maximal priority occurring
infinitely often would have to be both even and odd.

On the other hand, it is not obvious that every node should belong to either of $W_0$ or $W_1$. However, this
is indeed the case and known as \emph{determinacy}: a player has a strategy for a game iff the opponent does
not have a strategy for that game.

\begin{theorem}[\cite{Mart75,Gurevich-Harrington/82,focs91*368}]
Let $G = (V,V_0,V_1,E,\Omega)$ be a parity game. Then $W_0 \cap W_1 = \emptyset$ and $W_0 \cup W_1 = V$.
\end{theorem}

A strategy $\sigma$ for player $i$ is called \emph{positional} or \emph{memory-less} or \emph{history-free} if
for all $v_0\ldots v_n \in V^*V_i$ and all $w_0\ldots w_m \in V^*V_i$ we have: if $v_n = w_m$ then
$\sigma(v_0\ldots v_n) = \sigma(w_0\ldots w_m)$. That is, the value of the strategy on a finite path
only depends on the last node on that path. An important feature of parity games is the fact that such
strategies suffice.

\begin{theorem}[\cite{focs91*368}]
Let $G = (V,V_0,V_1,E,\Omega)$ be a parity game, $v \in V$, and $i \in \{0,1\}$. Player $i$ has a winning
strategy for $G$ starting in $v$ iff player $i$ has a positional winning strategy for $G$ starting in $v$.
\end{theorem}

A positional strategy $\sigma$ for player $i$ induces a \emph{subgame} 
$G|_\sigma := (V, V_0, V_1, E|_\sigma, \Omega)$ where 
$E|_\sigma := \{(u, v) \in E \mid u \in dom(\sigma) \Rightarrow \sigma(u) = v\}$. Such a subgame $G|_\sigma$ 
is, roughly speaking, basically the same game as $G$ with the restriction that whenever $\sigma$ provides a 
strategy decision for a node $u \in V_i$ all transitions from $u$ but $\sigma(u)$ are no longer accessible.

A set $U \subseteq V$ is said to be $i$-closed iff player $i$ can force any play to stay within $U$. This
means that player $1-i$ must not able to leave $U$ but player $i$ must always have the choice to remain 
inside $U$: 
\begin{displaymath}
\forall v \in U:\, \big(\ v \in V_{1-i}\, \Rightarrow \, vE \subseteq U\ \big)
\enspace \mbox{and} \enspace
\big(\ v \in V_i\, \Rightarrow \, vE \cap U \ne \emptyset\ \big)
\end{displaymath}
Note that $W_0$ is $0$-closed and $W_1$ is $1$-closed.

A set $U \subseteq V$ induces a \emph{subgame} 
$G|_U := (U, U \cap V_0, U \cap V_1, E \cap U \times U, \Omega|_U)$ iff the underlying transition relation 
$E \cap U \times U$ remains total i.e. for all $u \in U$ there is at least one $v \in U$ s.t.\ $uEv$. Clearly, 
each $i$-closed set $U$ induces a subgame. We often identify a set $U \subseteq V$ that induces a subgame 
w.r.t.\ a fixed parity game with the induced subgame itself.

\subsection{Dominions}

A set $U \subseteq V$ is called an \emph{$i$-dominion} iff $U$ is $i$-closed and the induced subgame is won by player $i$. Clearly, $W_0$ is a $0$-dominion and $W_1$ is a $1$-dominion. That is, an $i$-dominion $U$ covers the idea of a region in the game graph that is won by player $i$ by forcing player $1-i$ to stay in $U$ on the one hand; but on the other hand an $i$-dominion $U$ is only won by player $i$ when using a winning strategy on $U$.

To see more precisely what the concept of dominions is used for we need to introduce \emph{attractors} and 
\emph{SCC decompositions} of parity games.


\subsection{Attractors and Attractor Strategies}
Let $U \subseteq V$ and $i \in \{0,1\}$. Define for all $k \in \Nat$
\begin{align*}
\attr{0}{i}{U} \enspace := \enspace &U \\
\attr{k+1}{i}{U} \enspace := \enspace &\attr{k}{i}{U} \\
\cup\enspace &(V_i \cap \{ v \mid vE \cap \attr{k}{i}{U} \ne \emptyset \}) \\
\cup\enspace &(V_{1-i} \cap \{ v \mid vE \subseteq \attr{k}{i}{U} \}) \\
\attr{}{i}{U} \enspace := \enspace &\bigcup\limits_{k \in \Nat} \attr{k}{i}{U}
\end{align*}
Intuitively, $\attr{k}{i}{U}$ consists of all nodes s.t.\ player $i$ can force any play to reach $U$ in
at most $k$ moves. 
%Attractors are necessary in order to be able to use local solvers for the global problem.

\begin{lemma}[\cite{TCS::Zielonka1998,Stirling95}]
\label{lem:minusattr}
Let $G = (V,V_0,V_1,E,\Omega)$ be a parity game and $U \subseteq V$. Let $V' := V \setminus \attr{}{i}{U}$.
Then $G' = (V',V_0 \cap V',V_1 \cap V', E \cap V'\times V',\Omega)$ is again a parity game with its
underlying graph being total.
\end{lemma}

In other words $V \setminus \attr{}{i}{U}$ is $(1-i)$-closed; if additionally $U$ is an $i$-dominion then 
$\attr{}{i}{U}$ also is an $i$-dominion. This yields a general procedure for solving parity games: find a 
dominion in the game graph that is won by one of the two players, build its attractor of the dominion and 
investigate the complement subgame.

Each attractor for player $i$ induces an \emph{attractor strategy} for player $i$. It is defined for all
$v \in \attr{k}{i}{U} \cap V_i$ for any $k \ge 1$ as $\sigma(v) = w$ iff $w \in \attr{k-1}{i}{U}$.









%%% Local Variables:
%%% mode: latex
%%% TeX-master: "main"
%%% End:
