% !TEX root =  main.tex

%\section{Benchmark Games}

%The \pgsolver library contains a few programs that create benchmarks for the parity game
%solvers. They are built using the command
%\begin{verbatim}
%    ~/pgsolver> make generators
%\end{verbatim}
%and reside afterwards in the subdirectory \texttt{bin}.
%
%Additionally, it is possible to compile all generators directly into \pgsolver by setting
%the variable \texttt{LINKGENERATORS} to \texttt{YES} in the \texttt{./Config}-file. This
%has the advantage that the process of generating and solving a game is not slowed down by
%formatting the game first and parsing it afterwards again into the same data structure.

\section{Random Games}

\subsection{A Na\"{\i}ve Model}
Randomly generated parity games are the simplest form of benchmarking utility to assess the
performance of game solvers. The ones used here are parametrized by four parameters: the number
$n$ of nodes in the game, the highest priority $p$, a lower and an upper bound $l$ and $u$ on the
out-degree of each node. The call
\begin{verbatim}
    ~/pgsolver> bin/randomgame 1000 200 2 5
\end{verbatim}
creates a random game with 1000 nodes as follows: for each node $v$,
\begin{itemize}
\item the priority $\Omega(v)$ is uniformly chosen among $\{0,\ldots,200\}$;
\item node $v$ belongs to player $0$ with probability $0.5$ and therefore to player $1$ with
      probability $0.5$ as well;
\item a number $d$ is uniformly chosen in the range $\{2,\ldots,5\}$, and $d$ pairwise different
      nodes are uniformly chosen as $v$'s successors.
\end{itemize}
Beware that \texttt{bin/randomgame} may crash or not terminate if the lower bound on the outdegree
is greater than the upper bound or that one is greater than the number of nodes.


\subsection{Clustered Random Games}

Almost every game created by the random generator above consists of a single big SCC and many adhesive
chains.

Therefore this generator uses an other random model that accomplishes to bring a bit more structure into its instances. It depends on nine parameters: The number of nodes $n$, the highest possibly occurring priority $p$, an out-degree range $(l;h)$, a recursion depth $r$, a recursion breadth range $(a;b)$ and an interconnection range $(x;y)$. An instance of $\mathcal{G}^c_{(n,r)}$ with $c = (p,l,h,a,b,x,y)$ can be generated as follows. If $r = 0$ or $a > n$ simply return a randomly generated game with $n$ nodes, highest priority $p$ and out-degree bounds $l$ and $h$ as above. If $r > 0$ and $a \leq n$,
\begin{itemize}
\item a number $d$ is uniformly chosen in the range $\{a,\ldots,\min(b,n)\}$;
\item $d$ numbers $k_0,\ldots,k_{d-1}$ are uniformly chosen in the range $\{0,\ldots,n-1\}$ s.t.\ $\sum_{i=0}^{d-1} k_i = n$;
\item $d$ instances $G_0,\ldots,G_{d-1}$ are chosen, where $G_i$ is chosen from $\mathcal{G}^c_{(k_i,r-1)}$ (we assume the nodes of all $G_i$ to be pairwise different);
\item a game $G$ is created by the union of all $G_i$;
\item a number $e$ is uniformly chosen in the range $\{x,\ldots,y\}$;
\item $e$ additional uniformly chosen edges in $G$ are added to $G$;
\item the game $G$ is returned.
\end{itemize}

Basically, this recursive approach creates a topological tree structure that is partially intercepted by adding the additional interconnection edges.

A clustered random game with 1000 nodes, highest priority 2, out-degree bounds 2 and 5, recursion depth 3, recursion breadth between 4 and 6 and interconnection degree between 11 and 22 is created using the call
\begin{verbatim}
    ~/pgsolver> bin/clusteredrandomgame 1000 200 2 5 3 4 6 11 22
\end{verbatim}


\subsection{Steady Random Games}

The Steady Random Generator tries to circumvent many universal optimisations in order to particularly
benchmark the backend solvers. The Steady Random Generator is parametrized by five parameters: the number
$n$ of nodes (and different priorities) in the game, a lower and an upper bound and on the
out-degree of each node and a lower and an upper bound and on the in-degree of each node. The call
\begin{verbatim}
    ~/pgsolver> bin/steadygame 1000 2 4 3 5
\end{verbatim}
creates a random game with 1000 nodes as follows: for each node $v$,
\begin{itemize}
\item the priority $\Omega(v)$ is simply $v$;
\item node $v$ belongs to player $0$ with probability $0.5$ and therefore to player $1$ with
      probability $0.5$ as well;
\item a number $d$ is uniformly chosen in the range $\{2,\ldots,4\}$, and $d$ pairwise different
      nodes are uniformly chosen in $v$'s subgame as $v$'s successors;
\item a number $d$ is uniformly chosen in the range $\{3,\ldots,5\}$, and an attempt is made to ensure
      that $v$ has $d$ pairwise different predecessors.
\end{itemize}
All edges are iteratively determined as long as there are at least two different nodes with one of them violating
a lower bound and the other one being below the opposite upper bound uniformly between two such nodes.


\section{Special Graph Structures}

\subsection{Clique Games}

The clique game of order $n$ is $G_n = (\{0,\ldots,n-1\},\{0,2,4,\ldots\},\{1,3,5,\ldots\},E,\Omega)$
with $E = \{ (v,w) \mid v \ne w \}$ and $\Omega(v) = v$. It forms a clique in a directed graph without
self-cycles. Clique games exhibit a large amount of cycles in the game which may pose difficulties
for certain solvers. Note that adding self-cycles to the nodes will result in easy solving because
self-cycles are partial winning strategies in this case, and then all other nodes would lie in the attractor
of these winning regions.

The clique game of order $50$ for instance is generated as follows.
\begin{verbatim}
    ~/pgsolver> bin/cliquegame 50
\end{verbatim}
The clique game of order 50 with self-cycles can also be generated.
\begin{verbatim}
    ~/pgsolver> bin/cliquegame 50 self
\end{verbatim}


\subsection{Ladder Games}
\label{sec:laddergame}

The name of these games is derived from their structure which is reminiscent of a ladder. The
\emph{ladder game} of index $n$ is $G_n = (V,V_0,V_1,E,\Omega)$, defined as follows.
\begin{itemize}
\item $V = \{0,\ldots,2n-1\}$,
\item $V_0 = \{0,2,4,\ldots,2n-2\}$, $V_1 = \{1,3,5,\ldots,2n-1\}$,
\item $E = \{ (v,w) \mid w \equiv_{2n} v+i $ for some $i \in \{1,2\}$ $\}$,
\item $\Omega(v) = v \mod 2$.
\end{itemize}
where $\equiv_{2n}$ means equality modulo $2n$.

It is not hard to see that player $0$ wins on $V_0$ and player $1$ wins on $V_1$ since both can
stay within these regions and only see the priorities $0$, resp.\ $1$.

Ladder games are chosen as benchmarks because in $G_n$ there are $2^n$ many positional strategies for each
player but only one of them is a winning strategy. Not surprisingly, ladder games are very difficult
to solve for the strategy guessing heuristic for example.

The ladder game of index 19 is created using the call
\begin{verbatim}
    ~/pgsolver> bin/laddergame 19
\end{verbatim}


\section{Hard Games for Particular  Algorithms}


\subsection{Jurdzi{\'n}ski Games}
\label{sec:jurdgame}

Jurdzi{\'n}ski has defined a family of games on which the small progress measures algorithm
exhibits an exponential running time \cite{Jurdzinski/00}. The Jurdzi{\'n}ski game $J_{d,w}$ of depth 
$d$ and width $w$ forms a rectangle of dimension $(2d+1) \times (2w)$. For instance, $J_{50,100}$
is generated by calling
\begin{verbatim}
    ~/pgsolver> bin/jurdzinskigame 50 100
\end{verbatim}


% \begin{figure}[t]
% \begin{center}
% \scalebox{0.6}{\input{graphics/marcingame.eepic}}
% \end{center}

% \caption{The Jurdzi{\'n}ski games.}
% \label{fig:marcingame}
% \end{figure}



\subsection{Recursive Ladder Games}

%\newcommand{\playercirc}[3]{
	\pgfcircle[stroke]{\pgfpoint{#1}{#2}}{0.5cm}
	\pgfputat{\pgfpoint{#1}{#2}}{\pgfbox[center,center]{#3}}
}

\newcommand{\playerrect}[3]{
	\pgfrect[stroke]{\pgfrelative{\pgfpoint{#1}{#2}}{\pgfpoint{-0.5cm}{-0.5cm}}}{\pgfpoint{1.0cm}{1.0cm}}
	\pgfputat{\pgfpoint{#1}{#2}}{\pgfbox[center,center]{#3}}
}

\newcommand{\playercircs}[3]{
	\pgfcircle[stroke]{\pgfpoint{#1}{#2}}{0.25cm}
	\pgfputat{\pgfpoint{#1}{#2}}{\pgfbox[center,center]{#3}}
}

\newcommand{\playerrects}[3]{
	\pgfrect[stroke]{\pgfrelative{\pgfpoint{#1}{#2}}{\pgfpoint{-0.25cm}{-0.25cm}}}{\pgfpoint{0.5cm}{0.5cm}}
	\pgfputat{\pgfpoint{#1}{#2}}{\pgfbox[center,center]{#3}}
}

\newcommand{\playeredge}[2]{
	\pgfsetendarrow{\pgfarrowto}
	\pgfline{#1}{#2}
}

\newcommand{\playeredgecurve}[3]{
	\pgfsetendarrow{\pgfarrowto}
	\pgfmoveto{#1}
	\pgfcurveto{#2}{#2}{#3}
	\pgfstroke
}

\newcommand{\playeredgecurvex}[4]{
	\pgfsetendarrow{\pgfarrowto}
	\pgfmoveto{#1}
	\pgfcurveto{#2}{#3}{#4}
	\pgfstroke
}

\newcommand{\playeredgelr}[3]{
	\playeredge{\pgfrelative{\pgfpoint{#1}{#3}}{\pgfpoint{0.5cm}{0.0cm}}}{\pgfrelative{\pgfpoint{#2}{#3}}{\pgfpoint{-0.5cm}{0.0cm}}}
}

\newcommand{\playeredgerl}[3]{
	\playeredge{\pgfrelative{\pgfpoint{#1}{#3}}{\pgfpoint{-0.5cm}{0.0cm}}}{\pgfrelative{\pgfpoint{#2}{#3}}{\pgfpoint{0.5cm}{0.0cm}}}
}

\newcommand{\playeredgeud}[3]{
	\playeredge{\pgfrelative{\pgfpoint{#1}{#2}}{\pgfpoint{0cm}{-0.5cm}}}{\pgfrelative{\pgfpoint{#1}{#3}}{\pgfpoint{0cm}{0.5cm}}}
}

\newcommand{\playeredgedu}[3]{
	\playeredge{\pgfrelative{\pgfpoint{#1}{#2}}{\pgfpoint{0cm}{0.5cm}}}{\pgfrelative{\pgfpoint{#1}{#3}}{\pgfpoint{0cm}{-0.5cm}}}
}

\newcommand{\playeredgeblur}[4]{
	\playeredge{\pgfrelative{\pgfpoint{#1}{#2}}{\pgfpoint{0.35cm}{0.35cm}}}{\pgfrelative{\pgfpoint{#3}{#4}}{\pgfpoint{-0.35cm}{-0.35cm}}}
}

\newcommand{\playeredgeurbl}[4]{
	\playeredge{\pgfrelative{\pgfpoint{#1}{#2}}{\pgfpoint{-0.35cm}{-0.35cm}}}{\pgfrelative{\pgfpoint{#3}{#4}}{\pgfpoint{0.35cm}{0.35cm}}}
}

\newcommand{\playeredgeulbr}[4]{
	\playeredge{\pgfrelative{\pgfpoint{#1}{#2}}{\pgfpoint{0.35cm}{-0.35cm}}}{\pgfrelative{\pgfpoint{#3}{#4}}{\pgfpoint{-0.35cm}{0.35cm}}}
}

\newcommand{\playeredgebrul}[4]{
	\playeredge{\pgfrelative{\pgfpoint{#1}{#2}}{\pgfpoint{-0.35cm}{0.35cm}}}{\pgfrelative{\pgfpoint{#3}{#4}}{\pgfpoint{0.35cm}{-0.35cm}}}
}

\newcommand{\playeredgeselfdn}[2]{\playeredgecurvex{\pgfrelative{\pgfpoint{#1}{#2}}{\pgfpoint{0.16cm}{-0.5cm}}}{\pgfrelative{\pgfpoint{#1}{#2}}{\pgfpoint{0.60cm}{-1.40cm}}}{\pgfrelative{\pgfpoint{#1}{#2}}{\pgfpoint{-0.60cm}{-1.40cm}}}{\pgfrelative{\pgfpoint{#1}{#2}}{\pgfpoint{-0.16cm}{-0.50cm}}}}

\newcommand{\playeredgeselfrs}[2]{\playeredgecurvex{\pgfrelative{\pgfpoint{#1}{#2}}{\pgfpoint{+0.25cm}{0.08cm}}}{\pgfrelative{\pgfpoint{#1}{#2}}{\pgfpoint{+0.7cm}{+0.3cm}}}{\pgfrelative{\pgfpoint{#1}{#2}}{\pgfpoint{+0.7cm}{-0.3cm}}}{\pgfrelative{\pgfpoint{#1}{#2}}{\pgfpoint{+0.25cm}{-0.08cm}}}}

\newcommand{\playeredgecurvelr}[4]{\playeredgecurve{\pgfrelative{\pgfpoint{#1}{#4}}{\pgfpoint{0.5cm}{0.20cm}}}{\pgfrelative{\pgfpoint{#2}{#4}}{\pgfpoint{0cm}{0.5cm}}}{\pgfrelative{\pgfpoint{#3}{#4}}{\pgfpoint{-0.5cm}{0.20cm}}}}

\newcommand{\playeredgecurverl}[4]{\playeredgecurve{\pgfrelative{\pgfpoint{#1}{#4}}{\pgfpoint{-0.5cm}{-0.20cm}}}{\pgfrelative{\pgfpoint{#2}{#4}}{\pgfpoint{0cm}{-0.5cm}}}{\pgfrelative{\pgfpoint{#3}{#4}}{\pgfpoint{0.5cm}{-0.20cm}}}}

\newcommand{\playeredgecurvelrs}[4]{\playeredgecurve{\pgfrelative{\pgfpoint{#1}{#4}}{\pgfpoint{0.25cm}{0.10cm}}}{\pgfrelative{\pgfpoint{#2}{#4}}{\pgfpoint{0cm}{0.25cm}}}{\pgfrelative{\pgfpoint{#3}{#4}}{\pgfpoint{-0.25cm}{0.10cm}}}}

\newcommand{\playeredgecurverls}[4]{\playeredgecurve{\pgfrelative{\pgfpoint{#1}{#4}}{\pgfpoint{-0.25cm}{-0.10cm}}}{\pgfrelative{\pgfpoint{#2}{#4}}{\pgfpoint{0cm}{-0.25cm}}}{\pgfrelative{\pgfpoint{#3}{#4}}{\pgfpoint{0.25cm}{-0.10cm}}}}


\newcommand{\playeredgecurvetd}[4]{\playeredgecurve{\pgfrelative{\pgfpoint{#1}{#2}}{\pgfpoint{-0.20cm}{-.5cm}}}{\pgfrelative{\pgfpoint{#1}{#3}}{\pgfpoint{-.5cm}{0cm}}}{\pgfrelative{\pgfpoint{#1}{#4}}{\pgfpoint{-0.20cm}{.5cm}}}}

\newcommand{\playeredgecurvedt}[4]{\playeredgecurve{\pgfrelative{\pgfpoint{#1}{#2}}{\pgfpoint{0.20cm}{.5cm}}}{\pgfrelative{\pgfpoint{#1}{#3}}{\pgfpoint{.5cm}{0cm}}}{\pgfrelative{\pgfpoint{#1}{#4}}{\pgfpoint{0.20cm}{-.5cm}}}}

\newcommand{\pictrecursiveladder}[0]{
\begin{pgfpicture}{-1cm}{-1.5cm}{10cm}{9cm}
\pgfsetlinewidth{0.7pt}
\playerrect{0cm}{8cm}{$a_0: 1$}
\playercirc{0cm}{6cm}{$b_0: 1$}
\playerrect{0cm}{4cm}{$c_0: 5$}
\playercirc{0cm}{2cm}{$d_0: 4$}
\playerrect{0cm}{0cm}{$e_0: 3$}
\playercirc{3cm}{8cm}{$a_1: 0$}
\playerrect{3cm}{6cm}{$b_1: 0$}
\playercirc{3cm}{4cm}{$c_1: 8$}
\playerrect{3cm}{2cm}{$d_1: 7$}
\playercirc{3cm}{0cm}{$e_1: 6$}
\playerrect{6cm}{8cm}{$a_2: 1$}
\playercirc{6cm}{6cm}{$b_2: 1$}
\playerrect{6cm}{4cm}{$c_2: 11$}
\playercirc{6cm}{2cm}{$d_2: 10$}
\playerrect{6cm}{0cm}{$e_2: 9$}
\playercirc{9cm}{8cm}{$a_3: 0$}
\playerrect{9cm}{6cm}{$b_3: 0$}
\playeredgecurvelr{0cm}{1.5cm}{3cm}{2cm}
\playeredgecurverl{3cm}{1.5cm}{0cm}{2cm}
\playeredgecurvelr{3cm}{4.5cm}{6cm}{2cm}
\playeredgecurverl{6cm}{4.5cm}{3cm}{2cm}
\playeredgecurvetd{0cm}{8cm}{7cm}{6cm}
\playeredgecurvedt{0cm}{6cm}{7cm}{8cm}
\playeredgecurvetd{3cm}{8cm}{7cm}{6cm}
\playeredgecurvedt{3cm}{6cm}{7cm}{8cm}
\playeredgecurvetd{6cm}{8cm}{7cm}{6cm}
\playeredgecurvedt{6cm}{6cm}{7cm}{8cm}
\playeredgecurvetd{9cm}{8cm}{7cm}{6cm}
\playeredgecurvedt{9cm}{6cm}{7cm}{8cm}
\playeredgecurvetd{0cm}{2cm}{1cm}{0cm}
\playeredgecurvedt{0cm}{0cm}{1cm}{2cm}
\playeredgecurvetd{3cm}{2cm}{1cm}{0cm}
\playeredgecurvedt{3cm}{0cm}{1cm}{2cm}
\playeredgecurvetd{6cm}{2cm}{1cm}{0cm}
\playeredgecurvedt{6cm}{0cm}{1cm}{2cm}
\playeredgeud{0cm}{6cm}{4cm}
\playeredgeud{0cm}{4cm}{2cm}
\playeredgeud{3cm}{6cm}{4cm}
\playeredgeud{3cm}{4cm}{2cm}
\playeredgeud{6cm}{6cm}{4cm}
\playeredgeud{6cm}{4cm}{2cm}
\playeredge{\pgfpoint{0.5cm}{0cm}}{\pgfpoint{2.5cm}{5.9cm}}
\playeredge{\pgfpoint{3.5cm}{0cm}}{\pgfpoint{5.5cm}{5.9cm}}
\playeredge{\pgfpoint{6.5cm}{0cm}}{\pgfpoint{8.5cm}{5.9cm}}
\playeredge{\pgfpoint{0.5cm}{4cm}}{\pgfpoint{2.5cm}{6.1cm}}
\playeredge{\pgfpoint{3.5cm}{4cm}}{\pgfpoint{5.5cm}{6.1cm}}
\playeredge{\pgfpoint{6.5cm}{4cm}}{\pgfpoint{8.5cm}{6.1cm}}
\playeredge{\pgfpoint{2.5cm}{8cm}}{\pgfpoint{0.5cm}{2cm}}
\playeredge{\pgfpoint{5.5cm}{8cm}}{\pgfpoint{3.5cm}{2cm}}
\playeredge{\pgfpoint{8.5cm}{8cm}}{\pgfpoint{6.5cm}{2cm}}
\end{pgfpicture}
}




\newcommand{\thegame}[4]{
\newcommand{\declaneeven}{a}
\newcommand{\declaneodd}{b}
\newcommand{\declaneevenroot}{c}
\newcommand{\declaneoddroot}{d}
\newcommand{\cyclenodea}{e}
\newcommand{\cyclenodeb}{f}
\newcommand{\cyclenodec}{g}
\newcommand{\cyclecenter}{h}
\newcommand{\cycleaccess}{k}
\newcommand{\cycleselector}{l}
\newcommand{\cycleleaver}{m}
\newcommand{\upperselector}{z}
\newcommand{\finalsink}{p}
\newcommand{\finalcycle}{q}
\newcommand{\starteven}{s}
\newcommand{\startselector}{t}
\newcommand{\highentrybit}{w}
\newcommand{\lowestbit}{y}
\newcommand{\bitselector}{x}
\begin{pgfpicture}{#1}{#2}{#3}{#4}
\pgfsetlinewidth{0.7pt}
\playercirc{0cm}{2cm}{$\declaneodd_0: 21$}
\playerrect{2cm}{2cm}{$\declaneeven_0: 22$}
\playercirc{0cm}{4cm}{$\declaneodd_1: 23$}
\playerrect{2cm}{4cm}{$\declaneeven_1: 24$}
\playercirc{0cm}{6cm}{$\declaneodd_2: 25$}
\playerrect{2cm}{6cm}{$\declaneeven_2: 26$}
\playercirc{0cm}{8cm}{$\declaneodd_3: 27$}
\playerrect{2cm}{8cm}{$\declaneeven_3: 28$}
\playercirc{0cm}{10cm}{$\declaneodd_4: 29$}
\playerrect{2cm}{10cm}{$\declaneeven_4: 30$}
\playercirc{0cm}{12cm}{$\declaneodd_5: 31$}
\playerrect{2cm}{12cm}{$\declaneeven_5: 32$}
\playercirc{0cm}{14cm}{$\declaneodd_6: 33$}
\playerrect{2cm}{14cm}{$\declaneeven_6: 34$}
\playerrect{2cm}{0cm}{$\declaneevenroot: 35$}
\playercirc{0cm}{0cm}{$\declaneoddroot: 36$}
\playercirc{8cm}{8cm}{$\cyclenodea_0: 9$}
\playercirc{8cm}{14cm}{$\cyclenodea_1: 15$}
\playercirc{6cm}{6cm}{$\cyclenodeb_0: 11$}
\playercirc{6cm}{12cm}{$\cyclenodeb_1: 17$}
\playercirc{8cm}{4cm}{$\cyclenodec_0: 13$}
\playercirc{8cm}{10cm}{$\cyclenodec_1: 19$}
\playerrect{10cm}{6cm}{$\cyclecenter_0: 14$}
\playerrect{10cm}{12cm}{$\cyclecenter_1: 20$}
\playerrect{12cm}{4cm}{$\cycleaccess_0: 41$}
\playerrect{12cm}{10cm}{$\cycleaccess_1: 45$}
\playercirc{16cm}{4cm}{$\cycleselector_0: 10$}
\playercirc{16cm}{10cm}{$\cycleselector_1: 16$}
\playerrect{16cm}{6cm}{$\cycleleaver_0: 42$}
\playerrect{16cm}{12cm}{$\cycleleaver_1: 46$}
\playercirc{18cm}{6cm}{$\upperselector_0: 39$}
\playercirc{18cm}{12cm}{$\upperselector_1: 43$}
\playerrect{20cm}{6cm}{$\finalsink: 48$}
\playerrect{20cm}{4cm}{$\finalcycle: 1$}
\playercirc{13cm}{14cm}{$\lowestbit: 5$}
\playercirc{15cm}{14cm}{$\highentrybit: 7$}
\playercirc{-3cm}{8cm}{$\bitselector: 38$}
\playercirc{8cm}{0cm}{$\starteven: 2$}
\playercirc{-3cm}{6cm}{$\startselector: 3$}
\playeredgerl{2cm}{0cm}{0cm}
\playeredgerl{2cm}{0cm}{2cm}
\playeredgerl{2cm}{0cm}{4cm}
\playeredgerl{2cm}{0cm}{6cm}
\playeredgerl{2cm}{0cm}{8cm}
\playeredgerl{2cm}{0cm}{10cm}
\playeredgerl{2cm}{0cm}{12cm}
\playeredgerl{2cm}{0cm}{14cm}
\playeredgeud{0cm}{2cm}{0cm}
\playeredgeud{0cm}{4cm}{2cm}
\playeredgeud{0cm}{6cm}{4cm}
\playeredgeud{0cm}{8cm}{6cm}
\playeredgeud{0cm}{10cm}{8cm}
\playeredgeud{0cm}{12cm}{10cm}
\playeredgeud{0cm}{14cm}{12cm}
\playeredgelr{10cm}{16cm}{6cm}
\playeredgelr{10cm}{16cm}{12cm}
\playeredgelr{16cm}{18cm}{6cm}
\playeredgelr{16cm}{18cm}{12cm}
\playeredgerl{16cm}{12cm}{4cm}
\playeredgerl{16cm}{12cm}{10cm}
\playeredgeblur{16cm}{4cm}{18cm}{6cm}
\playeredgeblur{16cm}{10cm}{18cm}{12cm}
\playeredgelr{18cm}{20cm}{6cm}
\playeredgeulbr{18cm}{12cm}{20.35cm}{6.17cm}
\playeredgebrul{18cm}{6cm}{16cm}{10cm}
\playeredgeud{20cm}{6cm}{4cm}
\playeredgeselfdn{20cm}{4cm}
% \playeredgebrul{12cm}{4cm}{9.70cm}{5.83cm}
\playeredgebrul{11.87cm}{4.13cm}{9.70cm}{5.83cm}
%\playeredgebrul{12cm}{10cm}{9.70cm}{11.83cm}
\playeredgebrul{11.87cm}{10.13cm}{9.70cm}{11.83cm}
\playeredgebrul{10.35cm}{6.17cm}{8cm}{8cm}
\playeredgebrul{10.35cm}{12.17cm}{8cm}{14cm}
\playeredgeurbl{8cm}{8cm}{6cm}{6cm}
\playeredgeurbl{8cm}{14cm}{6cm}{12cm}
\playeredgeulbr{6cm}{6cm}{8cm}{4cm}
\playeredgeulbr{6cm}{12cm}{8cm}{10cm}
\playeredgeblur{8cm}{4cm}{10.30cm}{5.83cm}
\playeredgeblur{8cm}{10cm}{10.30cm}{11.83cm}
\playeredgecurvelr{13cm}{14cm}{15cm}{14cm}
\playeredgecurverl{15cm}{14cm}{13cm}{14cm}
\playeredgecurvex{\pgfpoint{13cm}{13.5cm}}{\pgfpoint{12cm}{7cm}}{\pgfpoint{15cm}{5cm}}{\pgfpoint{15.65cm}{4.35cm}}
\playeredgecurve{\pgfpoint{15cm}{13.5cm}}{\pgfpoint{14.5cm}{11cm}}{\pgfpoint{15.65cm}{10.35cm}}
\playeredgecurvex{\pgfpoint{15.5cm}{14cm}}{\pgfpoint{17cm}{14.5cm}}{\pgfpoint{19cm}{15cm}}{\pgfpoint{20.25cm}{6.5cm}}
\playeredgecurvex{\pgfpoint{6.45cm}{5.8cm}}{\pgfpoint{7.5cm}{6.5cm}}{\pgfpoint{10.5cm}{3.5cm}}{\pgfpoint{11.5cm}{3.8cm}}
\playeredgecurvex{\pgfpoint{6.45cm}{12.2cm}}{\pgfpoint{7.5cm}{12.5cm}}{\pgfpoint{10.5cm}{9.5cm}}{\pgfpoint{11.5cm}{10.2cm}}
\playeredgecurvex{\pgfpoint{6.45cm}{6.2cm}}{\pgfpoint{11cm}{8cm}}{\pgfpoint{10cm}{9cm}}{\pgfpoint{11.5cm}{9.8cm}}
\playeredgecurvex{\pgfpoint{6.45cm}{11.8cm}}{\pgfpoint{9cm}{12cm}}{\pgfpoint{11.5cm}{6.5cm}}{\pgfpoint{12cm}{4.5cm}}
\playeredgecurvex{\pgfpoint{8.5cm}{0cm}}{\pgfpoint{14cm}{4.5cm}}{\pgfpoint{17cm}{1.5cm}}{\pgfpoint{19.5cm}{5.9cm}}
% \playeredgeblur{8cm}{0cm}{12cm}{4cm}
\playeredgeblur{8cm}{0cm}{11.87cm}{3.87cm}
\playeredgecurve{\pgfpoint{8.45cm}{0.23cm}}{\pgfpoint{14cm}{4cm}}{\pgfpoint{12.45cm}{9.5cm}}
\playeredgecurve{\pgfpoint{8cm}{7.5cm}}{\pgfpoint{9.5cm}{4cm}}{\pgfpoint{8cm}{0.5cm}}
\playeredgecurvex{\pgfpoint{8cm}{13.5cm}}{\pgfpoint{9cm}{11cm}}{\pgfpoint{10.5cm}{3.5cm}}{\pgfpoint{8.2cm}{0.45cm}}
\playeredgecurvex{\pgfpoint{-0.25cm}{1.55cm}}{\pgfpoint{-0.1cm}{-0.6cm}}{\pgfpoint{-4cm}{-1.5cm}}{\pgfpoint{7.54cm}{-0.2cm}}
\playeredgecurvex{\pgfpoint{-0.25cm}{3.55cm}}{\pgfpoint{-0.9cm}{-0.8cm}}{\pgfpoint{-4cm}{-2.0cm}}{\pgfpoint{7.58cm}{-0.25cm}}
\playeredgecurvex{\pgfpoint{-0.25cm}{5.55cm}}{\pgfpoint{-1.7cm}{-1.0cm}}{\pgfpoint{-4cm}{-2.5cm}}{\pgfpoint{7.62cm}{-0.3cm}}
\playeredgecurvex{\pgfpoint{-0.25cm}{7.55cm}}{\pgfpoint{-2.5cm}{-1.2cm}}{\pgfpoint{-4cm}{-3.0cm}}{\pgfpoint{7.66cm}{-0.35cm}}
\playeredgecurvex{\pgfpoint{-0.25cm}{9.55cm}}{\pgfpoint{-3.3cm}{-1.4cm}}{\pgfpoint{-4cm}{-3.5cm}}{\pgfpoint{7.70cm}{-0.40cm}}
\playeredgecurvex{\pgfpoint{-0.25cm}{11.55cm}}{\pgfpoint{-4.1cm}{-1.6cm}}{\pgfpoint{-4cm}{-4.0cm}}{\pgfpoint{7.75cm}{-0.45cm}}
\playeredgecurvex{\pgfpoint{-0.25cm}{13.55cm}}{\pgfpoint{-4.9cm}{-1.8cm}}{\pgfpoint{-4cm}{-4.5cm}}{\pgfpoint{7.80cm}{-0.50cm}}
\playeredge{\pgfpoint{-0.3cm}{0.4cm}}{\pgfpoint{-2.71cm}{7.55cm}}
\playeredge{\pgfpoint{-0.5cm}{2cm}}{\pgfpoint{-2.64cm}{7.64cm}}
\playeredge{\pgfpoint{-0.5cm}{4cm}}{\pgfpoint{-2.58cm}{7.76cm}}
\playeredge{\pgfpoint{-0.5cm}{6cm}}{\pgfpoint{-2.53cm}{7.88cm}}
\playeredge{\pgfpoint{-0.5cm}{8cm}}{\pgfpoint{-2.5cm}{8cm}}
\playeredge{\pgfpoint{-0.5cm}{10cm}}{\pgfpoint{-2.53cm}{8.12cm}}
\playeredge{\pgfpoint{-0.5cm}{12cm}}{\pgfpoint{-2.58cm}{8.24cm}}
\playeredge{\pgfpoint{-0.5cm}{14cm}}{\pgfpoint{-2.64cm}{8.36cm}}
\playeredgedu{-3cm}{6cm}{8cm}
\playeredge{\pgfpoint{-0.5cm}{0cm}}{\pgfpoint{-2.7cm}{5.6cm}}
\playeredgecurvex{\pgfpoint{-3cm}{5.5cm}}{\pgfpoint{-4cm}{-1.8cm}}{\pgfpoint{-2cm}{-4.5cm}}{\pgfpoint{8cm}{-0.5cm}}
\playeredgecurvex{\pgfpoint{-3cm}{8.5cm}}{\pgfpoint{-4cm}{19cm}}{\pgfpoint{8cm}{15cm}}{\pgfpoint{14.7cm}{14.4cm}}
\playeredgecurvex{\pgfpoint{-2.9cm}{8.48cm}}{\pgfpoint{-4cm}{18.5cm}}{\pgfpoint{8cm}{15cm}}{\pgfpoint{12.7cm}{14.4cm}}
\playeredgecurvex{\pgfpoint{7.53cm}{14.1cm}}{\pgfpoint{5cm}{14.2cm}}{\pgfpoint{4.5cm}{11.9cm}}{\pgfpoint{2.5cm}{12cm}}
\playeredgecurvex{\pgfpoint{7.53cm}{10.1cm}}{\pgfpoint{5cm}{10.0cm}}{\pgfpoint{4.5cm}{14.1cm}}{\pgfpoint{2.5cm}{14cm}}
\playeredgecurvex{\pgfpoint{5.53cm}{12.1cm}}{\pgfpoint{4.5cm}{12.2cm}}{\pgfpoint{3.5cm}{9.9cm}}{\pgfpoint{2.5cm}{10cm}}
\playeredgecurvex{\pgfpoint{7.5cm}{10cm}}{\pgfpoint{5cm}{10.2cm}}{\pgfpoint{4.5cm}{7.9cm}}{\pgfpoint{2.5cm}{8.1cm}}
\playeredgecurvex{\pgfpoint{7.53cm}{4.1cm}}{\pgfpoint{5cm}{4.0cm}}{\pgfpoint{4.5cm}{8.1cm}}{\pgfpoint{2.5cm}{7.9cm}}
\playeredgecurvex{\pgfpoint{7.5cm}{14.0cm}}{\pgfpoint{3.5cm}{14.2cm}}{\pgfpoint{4.0cm}{5.9cm}}{\pgfpoint{2.5cm}{6.1cm}}
\playeredgecurvex{\pgfpoint{7.53cm}{8.1cm}}{\pgfpoint{5cm}{8.2cm}}{\pgfpoint{4.5cm}{5.9cm}}{\pgfpoint{2.5cm}{5.9cm}}
\playeredgecurvex{\pgfpoint{6cm}{11.5cm}}{\pgfpoint{6cm}{11cm}}{\pgfpoint{5.0cm}{5cm}}{\pgfpoint{2.5cm}{4.1cm}}
\playeredgecurvex{\pgfpoint{5.5cm}{6cm}}{\pgfpoint{4.5cm}{6.2cm}}{\pgfpoint{3.5cm}{3.9cm}}{\pgfpoint{2.5cm}{3.9cm}}
\playeredgecurvex{\pgfpoint{7.53cm}{9.9cm}}{\pgfpoint{3.5cm}{10.2cm}}{\pgfpoint{4.0cm}{1.9cm}}{\pgfpoint{2.5cm}{2.1cm}}
\playeredgecurvex{\pgfpoint{7.53cm}{3.9cm}}{\pgfpoint{5cm}{4.2cm}}{\pgfpoint{4.5cm}{1.9cm}}{\pgfpoint{2.5cm}{1.9cm}}
\playeredgecurvex{\pgfpoint{7.53cm}{13.9cm}}{\pgfpoint{3.0cm}{14.2cm}}{\pgfpoint{3.5cm}{-0.1cm}}{\pgfpoint{2.5cm}{0.1cm}}
\playeredgecurvex{\pgfpoint{7.53cm}{7.9cm}}{\pgfpoint{3.5cm}{8.2cm}}{\pgfpoint{4.0cm}{-0.1cm}}{\pgfpoint{2.5cm}{-0.1cm}}
\end{pgfpicture}
}

\newcommand{\pictexpstratimpr}{\thegame{0cm}{-1.5cm}{22cm}{14cm}}


\newcommand{\pictmodelcheckerladder}{
\begin{pgfpicture}{-2cm}{-2cm}{12cm}{4cm}
\pgfsetlinewidth{0.7pt}
\playercirc{0cm}{2cm}{$1$}
\playerrect{2cm}{2cm}{$2$}
\playercirc{4cm}{2cm}{$3$}
\playerrect{6cm}{2cm}{$4$}
\playercirc{8cm}{2cm}{$5$}
\playerrect{10cm}{2cm}{$6$}
\playercirc{0cm}{0cm}{$12$}
\playerrect{2cm}{0cm}{$11$}
\playercirc{4cm}{0cm}{$10$}
\playerrect{6cm}{0cm}{$9$}
\playercirc{8cm}{0cm}{$8$}
\playerrect{10cm}{0cm}{$7$}
\playeredgelr{0cm}{2cm}{2cm}
\playeredgelr{2cm}{4cm}{2cm}
\playeredgelr{4cm}{6cm}{2cm}
\playeredgelr{6cm}{8cm}{2cm}
\playeredgelr{8cm}{10cm}{2cm}
\playeredgeud{0cm}{2cm}{0cm}
\playeredgeud{2cm}{2cm}{0cm}
\playeredgeud{4cm}{2cm}{0cm}
\playeredgeud{6cm}{2cm}{0cm}
\playeredgeud{8cm}{2cm}{0cm}
\playeredgeud{10cm}{2cm}{0cm}
\playeredgeblur{0cm}{0cm}{1.85cm}{1.85cm}
\playeredgeblur{2.15cm}{0.15cm}{4cm}{2cm}
\playeredgeblur{4cm}{0cm}{5.85cm}{1.85cm}
\playeredgeblur{6.15cm}{0.15cm}{8cm}{2cm}
\playeredgeblur{8cm}{0cm}{9.85cm}{1.85cm}
\playeredgeud{0cm}{3.5cm}{2cm}
\playeredgelr{-1.5cm}{0cm}{2cm}
\pgfsetendarrow{}
\pgfline{\pgfpoint{10cm}{2.5cm}}{\pgfpoint{10cm}{3cm}}
\pgfline{\pgfpoint{10cm}{3cm}}{\pgfpoint{0cm}{3cm}}
\pgfline{\pgfpoint{10cm}{-0.5cm}}{\pgfpoint{10cm}{-1cm}}
\pgfline{\pgfpoint{10cm}{-1cm}}{\pgfpoint{-1cm}{-1cm}}
\pgfline{\pgfpoint{-1cm}{-1cm}}{\pgfpoint{-1cm}{2cm}}
\end{pgfpicture}
}

% \begin{figure}[t]
% \begin{center}
% \pictrecursiveladder
% \end{center}
% \caption{The recursive ladder game of index 4.}
% \label{fig:recursiveladder}
% \end{figure}

The recursive ladder games form a family on which the recursive algorithm exhibits an exponential
running time.  The recursive ladder game of index $n$ has size $5 \cdot n$. 
% We omit a formal
% definition here. The construction can easily be inferred from Fig.~\ref{fig:recursiveladder}. Inside
% the nodes it shows the node's identifier (left of the colon) and its priority.
The recursive ladder game of index 4 results in 20 nodes and is generated by calling
\begin{verbatim}
    ~/pgsolver> bin/recursiveladder 4
\end{verbatim}





\subsection{Exponential Strategy Improvement Games}

The exponential strategy improvement games form a family on which both strategy improvement algorithms
exhibit an exponential running time (in the index of the game) \cite{FriedmannSI09}.
The exponential strategy improvement game of index $n$ has size $14 \cdot n + 11$. 
% We omit a formal definition here. The game of index 2 is depicted in Fig.~\ref{fig:expstratimpr}.
% Inside the nodes it
%shows the node's identifier (left of the colon) and its priority.
A game of index 4 results in 67 nodes and is generated by calling
\begin{verbatim}
    ~/pgsolver> bin/expstratimpr 4
\end{verbatim}

In order to verify that the strategy iteration indeed requires exponential runtime on these games, you
need to either disable all generic optimisations or transform the game using the \texttt{transformer}
tool with the parameters \texttt{-ss -ap -bn}.


% \begin{landscape}
% % \pagenumbering{none}
% \begin{figure}[t]
% \begin{center}
% \pictexpstratimpr
% \end{center}
% \caption{The exponential strategy improvement game of index 2.}
% \label{fig:expstratimpr}
% \end{figure}
% \end{landscape}



\subsection{Model Checker Ladder Games}

The model checker ladder games form a family on which the model checker algorithm exhibits an
exponential running time.  The model checker ladder game of index $n$ has size $4 \cdot n$. 
% We omit a formal definition here. The construction can easily be inferred from
% Fig.~\ref{fig:modelcheckerladder}.
%Inside the nodes it shows the node's identifier (left of the colon) and its priority.
A model checker ladder game of index 3 results in 12 nodes and is generated by calling
\begin{verbatim}
    ~/pgsolver> bin/modelcheckerladder 3
\end{verbatim}

% \begin{figure}[h]
% \begin{center}
% \pictmodelcheckerladder
% \end{center}
% \caption{The model checker ladder game of index 3.}
% \label{fig:modelcheckerladder}
% \end{figure}


\section{Verification Problems}

\subsection{Fairness of an Elevator System}

We encode a simple fairness verification problem as a parity game. States of a transition system modelling
an \emph{elevator} for $n$ floors are of type
$\{1,\ldots,n\} \times \{\mathtt{o},\mathtt{c}\} \times (\bigcup \{ \mathit{Perm}(S) \mid S \subseteq \{1,\ldots,n\})$.
The first component describes the current position of the elevator as one of the floors. The
second component indicates whether the door is \emph{open} or \emph{closed}. The third component -- a permutation
of a subset of all available floors -- holds the \emph{requests}, i.e.\ those floors that should be served next.
The transitions on these are as follows.
\begin{itemize}
\item At any moment, any request or none can be issued. For simplicity reasons, we assume that at most one floor
      is added to the requests per transition. Note that nondeterministically, no request can be issued, and a
      request for a certain floor that is already contained in the current requests does not change them.
\item If the door is open then it is closed in the next step, the current floor does not change.
\item If it is closed, the elevator moves one floor (up or down) into the direction of the first request. If the
      floor reached that way is among the requested ones, the door is opened and that floor is removed from the
      current requests. Otherwise, the door remains closed.
\end{itemize}
We consider two different implementations of this elevator model: the first one stores requests in FIFO
style, the second in LIFO style. The games $G_n$ (with FIFO), resp.\ $G'_n$ (with LIFO) result from
encoding the model checking problem for this transition system and the CTL$^*$ formula
$\mathtt{A}(\mathtt{GF}\mathit{isPressed} \to \mathtt{GF}\mathit{isAt})$ as a parity game
\cite{Stirling95}.  Proposition $\mathit{isPressed}$ holds in any state s.t.\ the request list contains
the number $n$, and $\mathit{isAt}$ holds in a state where the current floor is $n$. Hence, this
formula requires all runs of the elevator to satisfy the following fairness property: if the top floor
is requested infinitely often then it is being served infinitely often. It can easily be formulated in
the modal $\mu$-calculus using a formula of size $11$ and alternation depth $2$ (of type
$\nu$--$\mu$--$\nu$). Hence the resulting parity games have constant index 3.  Note that $G_n$
encodes a positive instance of the model checking problem whereas $G'_n$ encodes a negative one.
The game $G_4$ is generated by calling
\begin{verbatim}
    ~/pgsolver> bin/elevators 4
\end{verbatim}
and the game $G'_5$ by calling
\begin{verbatim}
    ~/pgsolver> bin/elevators -u 5
\end{verbatim}

\subsection{Synthesising a Traffic Light Controller}

Consider the following problem. Road work is blocking one line of traffic on a two-lined road. There are two traffic lights
controlling the flow of traffic into the narrow part from each side. We want to find a strategy for switching the traffic lights
which ensures that 
\begin{itemize}
\item no car can enter the narrow part of road if there is an approaching car in this part already,
\item every car that appears at either end of the road works will eventually be able to pass.
\end{itemize}
This synthesis problem can be modelled as a simple model checking problem for the modal $\mu$-calculus and, hence, as a parity 
game problem. Consider the transition system $\mathcal{R}_n$ with states of the form 
\begin{displaymath}
(r_0,b_{\mathsf{left}},r_1,r_2\ldots,r_{n-2},b_{\mathsf{right}},r_{n-1},t) 
\end{displaymath}
where 
\begin{itemize}
\item $b_{\mathsf{left}},b_{\mathsf{right}} \in \{ \mathsf{red},\mathsf{green} \}$ model the two traffic lights,
\item $r_0,\ldots,r_{n-1} \in \{\mathsf{emp},\mathsf{carL},\mathsf{carR}\}$ model segments of the road which, at any moment,
      are either empty or occupied by a car driving into a particular direction, and   
\item $t \in \{\mathsf{trLght},\mathsf{env}\}$ is used to indicate whether the next change of state can be 
      a switching of the traffic light or a moving of the cars (seen as the \emph{environment} acting against the traffic light). 
\end{itemize} 
Transitions, labeled with actions, are as follows. First, both traffic light and environment can always choose not to do anything.
\begin{displaymath}
\Transition{(r_0,b_{\mathsf{left}},r_1,r_2\ldots,r_{n-2},b_{\mathsf{right}},r_{n-1},t)}{\mathsf{stay}}{(r_0,b_{\mathsf{left}},r_1,r_2\ldots,r_{n-2},b_{\mathsf{right}},r_{n-1},\overline{t})} 
\end{displaymath}
where $\overline{t}$ denotes the actor that is not $t$, i.e.\ $\overline{\mathsf{trLght}} = \mathsf{env}$ etc.

When it is the traffic lights' turn, either light can change colour:
\begin{align*}
&\Transition{(r_0,b_{\mathsf{left}},\ldots,b_{\mathsf{right}},r_{n-1},\mathsf{trLght})}{\mathsf{swLght}}{(r_0,\overline{b_{\mathsf{left}}},\ldots,b_{\mathsf{right}},r_{n-1},\mathsf{env})} \\
&\Transition{(r_0,b_{\mathsf{left}},\ldots,b_{\mathsf{right}},r_{n-1},\mathsf{trLght})}{\mathsf{swLght}}{(r_0,b_{\mathsf{left}},\ldots,\overline{b_{\mathsf{right}}},r_{n-1},\mathsf{env})} 
\end{align*}
where $\overline{\mathsf{red}} = \mathsf{green}$ etc.

Finally, when it is the environment's turn, any car that is already present can move forwards and free empty spaces at the sides can
be occupied by cars, modelled by transitions of the form
\begin{displaymath}
\Transition{(r_0,b_{\mathsf{left}},\ldots,b_{\mathsf{right}},r_{n-1},\mathsf{env})}{\mathsf{move}}{(r'_0,b_{\mathsf{left}},\ldots,b_{\mathsf{right}},r'_{n-1},\mathsf{trLght})}
\end{displaymath}
such that one of the following cases holds.
\begin{itemize}
\item $r_0 = \mathsf{emp}$, $r'_0 = \mathsf{carR}$
\item $r_0 = \mathsf{carR} = r'_1$, $b_{\mathsf{left}} = \mathsf{green}$, $r_1 = \mathsf{emp} = r'_0$ 
\item $\exists i$: $1 \le i \le n-3$, $r_i = \mathsf{carR} = r'_{i+1}$, $r_{i+1} = \mathsf{emp} = r'_i$  
\item $r_{n-2} = \mathsf{carR}$, $r'_{n-2} = \mathsf{emp}$ 
\item $r_{n-1} = \mathsf{emp}$, $r'_{n-1} = \mathsf{carL}$
\item $r_{n-1} = \mathsf{carL} = r'_{n-2}$, $b_{\mathsf{right}} = \mathsf{green}$, $r_{n-2} = \mathsf{emp} = r'_{n-1}$ 
\item $\exists i$: $2 \le i \le n-2$, $r_i = \mathsf{carL} = r'_{i-1}$, $r_{i-1} = \mathsf{emp} = r'_i$  
\item $r_1 = \mathsf{carL}$, $r'_1 = \mathsf{emp}$ 
\end{itemize}
In all cases, all unmentioned variables remain the same, i.e.\ we have $r'_j = r_j$ for those $j$ that are unaffected by these
conditions.

We use the following propositions to model elementary properties of such states of the form
$(r_0,b_{\mathsf{left}},r_1,\ldots,r_{n-2},b_{\mathsf{right}},r_{n-1},t)$.
\begin{itemize}
\item $\mathsf{green}_{\mathsf{left}}$ and $\mathsf{green}_{\mathsf{right}}$ hold in states in which the left, resp., right traffic
      light is set to green, i.e.\ $b_{\mathsf{left}} = \mathsf{green}$ etc.;
\item $\mathsf{wait}_{\mathsf{left}}$ and $\mathsf{wait}_{\mathsf{right}}$ hold in states in which the leftmost, resp.\ rightmost
      road segment, i.e.\ those that are guarded by the respective traffic lights, are occupied by a car: $r_0 = \mathsf{carR}$ or
      $r_{n-1} = \mathsf{carL}$;
\item $\mathsf{crash}$ holds when a car is given a green light despite the fact that the oncoming traffic has not passed yet: 
      \begin{align*}
        &(r_0 = \mathsf{carR} \wedge b_{\mathsf{left}} = \mathsf{green} \wedge \exists i. 1 \le i \le n-2 \wedge r_i = \mathsf{carL}) \\
        \vee & (r_{n-1} = \mathsf{carL} \wedge b_{\mathsf{right}} = \mathsf{green} \wedge \exists i. 1 \le i \le n-2 \wedge r_i = \mathsf{carR})
      \end{align*}
\end{itemize}
The existence of a controller for the traffic lights, ensuring that no car is allowed to enter the narrow part of the road when
a car facing the opposite direction is in it, as well as absence of infinite waiting for the cars is expressed by the following 
formula of the modal $\mu$-calculus.
\begin{align*}
\nu X_2.\mu X_1.\nu X_0. \neg\mathsf{crash} \wedge \Big(&\Mudiam{-}(\Mubox{\mathsf{stay}}X_0 \wedge \Mubox{\mathsf{move}}X_1)\ \vee \ \\
&\big( (\mathsf{green}_{\mathsf{left}} \vee \neg\mathsf{wait}_{\mathsf{left}}) \wedge
(\mathsf{green}_{\mathsf{right}} \vee \neg\mathsf{wait}_{\mathsf{right}}) \wedge \Mudiam{-}\Mubox{-}X_2\big)\Big)
\end{align*}
The parity game arising from the product of the road works system $\mathcal{R}_5$ for instance (with $5$ road segments) and the formula above can
be generated by calling
\begin{verbatim}
    ~/pgsolver> bin/roadworks 5
\end{verbatim}
This can be generalised to arbitrary parameters $n \ge 2$. The benchmark family can also be used directly in \pgsolver, for instance
for the road works system $\mathcal{R}_8$, by typing  
\begin{verbatim}
    ~/pgsolver> bin/pgsolver -gen roadworksvergm "8" -local stratimprloc2 0
\end{verbatim}
Note that node $0$ represents the model checking problem for the formula above and $\mathcal{R}_n$'s initial state in which both lights
are set to $\mathsf{red}$ and no cars are present.



\section{Particular Problems Encoded as Parity Games}

\subsection{Towers of Hanoi}

The Towers of Hanoi problem can be seen as a reachability game in the graph of game configurations.
The problem of size $n$ has $n$ disks that are to be moved from the first to the third rod.
A problem of size $3$ is generated by calling
\begin{verbatim}
    ~/pgsolver> bin/towersofhanoi 3
\end{verbatim}


\subsection{Language Inclusion Between an NBA and a DBA}

Let $n,m \ge 2$, $\Sigma_n = \{a_1,\ldots,a_n\}$ and consider the language $L_{n,m}$ of all words over the alphabet
$\Sigma_n$ that contain infinitely many occurrences of $(a_i)^m$ for some $i \{ 1,\ldots,n\}$. This language is 
accepted by the following NBA $\mathcal{A}_{n,m}$.
\begin{center}
\begin{tikzpicture}[thick,->,initial text={},every state/.style={shape=ellipse}, node distance=2.3cm]
  \node[state,initial]   (0)                    {$0$};
  \node[state]           (1) at ($(0)+(2,2)$)   {$1$};
  \node[state]           (n1) [right of=1]      {$n{+}1$};
  \node[state]           (2n1) [right of=n1]    {$2n{+}1$};
  \node                  (d1) [right of=2n1]    {$\ldots$};
  \node[state,accepting] (m2n1) [right of=d1]   {$(m{-}2)n{+}1$};
  \node[state]           (2) at ($(0)+(2,.5)$) {$2$};
  \node[state]           (n2) [right of=2]      {$n{+}2$};
  \node[state]           (2n2) [right of=n2]    {$2n{+}2$};
  \node                  (d2) [right of=2n2]    {$\ldots$};
  \node[state,accepting] (m2n2) [right of=d2]   {$(m{-}2)n{+}2$};
  \node[state]           (n) at ($(0)+(2,-2)$)  {$n$};
  \node[state]           (2n) [right of=n]      {$2n$};
  \node[state]           (3n) [right of=2n]     {$3n$};
  \node                  (dn) [right of=3n]    {$\ldots$};
  \node[state,accepting] (m1n) [right of=dn]   {$(m{-}1)n$};

  \node (ds1) at ($(0)+(2,-.5)$) {$\vdots$};
  \node (ds2) [right of=ds1] {$\vdots$};
  \node (ds3) [right of=ds2] {$\vdots$};
  \node (ds4) [right of=ds3] {};
  \node ()    [right of=ds4] {$\vdots$};

  \path (0)    edge [out=145,in=125,loop] node [above]                   {$\Sigma_n$} ()
               edge [out=90,in=180]     node [above,sloped]            {$a_1$}      (1)
               edge              node [above,sloped]            {$a_2$}      (2)
               edge [out=270,in=180]             node [above,sloped]            {$a_n$}      (n)
        (1)    edge              node [above]                   {$a_1$}      (n1)
        (n1)   edge              node [above]                   {$a_1$}      (2n1)
        (2n1)  edge              node [above]                   {$a_1$}      (d1)
        (d1)   edge              node [above]                   {$a_1$}      (m2n1)
        (2)    edge              node [above]                   {$a_2$}      (n2)
        (n2)   edge              node [above]                   {$a_2$}      (2n2)
        (2n2)  edge              node [above]                   {$a_2$}      (d2)
        (d2)   edge              node [above]                   {$a_2$}      (m2n2)
        (m2n2) edge [bend left=10]  node [very near start,sloped,below] {$a_2$}      (0)
        (n)    edge              node [above]                   {$a_n$}      (2n)
        (2n)   edge              node [above]                   {$a_n$}      (3n)
        (3n)   edge              node [above]                   {$a_n$}      (dn)
        (dn)   edge              node [below]                   {$a_n$}      (m1n);
%        (m1n)  edge              node [near start,sloped,above] {$a_n$}      (0);

  \draw[->] (m2n1) .. controls ($(m2n1)+(-2.3,-1.4)$) and ($(0)+(1.5,2.2)$) .. (0) node [near start,above] {$a_1$};
  \draw[->] (m1n) .. controls ($(m1n)+(-2.3,1)$) and ($(0)+(2,-2)$) .. (0) node [near start,above] {$a_n$};
\end{tikzpicture}
\end{center}
It is also accepted by the following deterministic B\"uchi automata $\mathcal{B}_{n,m}$.
\begin{center}
\begin{tikzpicture}[thick,->,node distance=2.3cm,initial text={},every state/.style={shape=ellipse}]
  \node[state,initial,accepting] (0)                        {$0$};
  \node[state,accepting]         (1)      [below of=0]      {$1$};
  \node                          (d1)     [below of=1,node distance=1.5cm]      {$\vdots$}; 
  \node[state,accepting]         (nm1)    [below of=d1,node distance=1.5cm]     {$n{-}1$};
  \node[state]                   (n)      [right of=0]      {$n$};
  \node[state]                   (np1)    [below of=n]      {$n{+}1$};
  \node                          (d2)     [below of=np1,node distance=1.5cm]    {$\vdots$}; 
  \node[state]                   (2nm1)   [below of=d2,node distance=1.5cm]     {$2n{-}1$};
  \node                          (d3)     [right of=n]      {$\ldots$};
  \node                          (d4)     [right of=np1]    {$\ldots$};
  \node                          (d5)     [right of=d2]     {$\vdots$};
  \node                          (d6)     [right of=2nm1]   {$\ldots$};
  \node[state]                   (mm1n)   [right of=d3]     {$(m{-}1)n$};
  \node[state]                   (mm1np1) [below of=mm1n]   {$(m{-}1)n{+}1$};
  \node                          (d7)     [below of=mm1np1,node distance=1.5cm] {$\vdots$}; 
  \node[state]                   (mnm1)   [below of=d7,node distance=1.5cm]     {$mn{-}1$};

  \path[->] (0)    edge node [above] {$a_1$} (n)
            (n)    edge node [above] {$a_1$} (d3)
            (d3)   edge node [above] {$a_1$} (mm1n)
            (1)    edge node [below] {$a_2$} (np1)
            (np1)  edge node [above] {$a_2$} (d4)
            (d4)   edge node [above] {$a_2$} (mm1np1)
            (nm1)  edge node [above] {$a_n$} (2nm1)
            (2nm1) edge node [above] {$a_n$} (d6)
            (d6)   edge node [above] {$a_n$} (mnm1);

  \path[->] (0)      edge                       node [sloped,above,near start]      {$a_2$} (np1)
            (n)      edge [bend right]          node [sloped,below]                 {$a_2$} (np1)
            (mm1n)   edge                       node [sloped,above,near start]      {$a_2$} (np1)
            (1)      edge                       node [sloped,above,near end]        {$a_1$} (n)
            (np1)    edge [bend right]          node [sloped,below]                 {$a_1$} (n)
            (mm1np1) edge                       node [sloped,above,near start]      {$a_1$} (n)
            (0)      edge                       node [sloped,above,near end]        {$a_n$} (2nm1)
            (n)      edge [bend right]          node [sloped,above,near start]        {$a_n$} (2nm1)
            (mm1n)   edge [bend right=5,in=170] node [sloped,above,very near start] {$a_n$} (2nm1)
            (nm1)    edge                       node [sloped,above,near start]      {$a_1$} (n)
            (2nm1)   edge [bend right]          node [sloped,below,near start]      {$a_1$} (n)
            (mnm1)   edge [in=-30,out=130]      node [sloped,above,very near start] {$a_1$} (n)
            (1)      edge                       node [sloped,below,near start]      {$a_n$} (2nm1)
            (np1)    edge [bend right]          node [sloped,below,near start]        {$a_n$} (2nm1)
            (mm1np1) edge                       node [sloped,above,near start]      {$a_n$} (2nm1)
            (nm1)    edge [in=220]              node [sloped,above,near start]      {$a_2$} (np1)
            (2nm1)   edge [bend right]          node [sloped,below]                 {$a_2$} (np1)
            (mnm1)   edge                       node [sloped,above,near start]      {$a_2$} (np1);
            
  \draw[->] (mm1n) .. controls ($(mm1n)+(-.3,1)$) and ($(0)+(.3,1)$) .. (0) node [near start,above] {$a_1$};
  \draw[->] (mm1np1) .. controls ($(mm1np1)+(-.3,-1)$) and ($(1)+(.3,-1)$) .. (1) node [near start,above] {$a_2$};
  \draw[->] (mnm1) .. controls ($(mnm1)+(-.3,-1)$) and ($(nm1)+(.3,-1)$) .. (nm1) node [near start,above] {$a_n$};
\end{tikzpicture}
\end{center}
The synchronous product of $\mathcal{A}_{n,m}$ and $\mathcal{B}_{n,m}$ is a parity game $\mathcal{G}_{n,m}$ in which
every node is owned by player $1$, and the priority of a node $(q,p)$ is given as
\begin{displaymath}
\Omega(q,p) \enspace = \enspace \begin{cases}
2 &,\text{if } p \text{ is accepting,} \\
1 &,\text{if } p \text{ is not accepting and } q \text{ is accepting,} \\
0 &,\text{otherwise.}
\end{cases}
\end{displaymath}
This is the standard encoding of fair simulation as a parity condition \cite{HenzingerKR02,HutagalungLL:LATA13}. This game
encodes the language inclusion problem between an NBA and a DBA. In this case, the game is entirely won by player $0$.

The game $\mathcal{G}_{6,9}$ is generated by
\begin{verbatim}
    ~/pgsolver> bin/langincl 6 9
\end{verbatim}
The second parameter is optional. If it is not given then $m=2$ is chosen by default.


 
%%% Local Variables:
%%% mode: latex
%%% TeX-master: "main"
%%% End:
